\documentclass[russian,utf8,nocolumnxxxi,nocolumnxxxii]{eskdtext}
\usepackage[T1,T2A]{fontenc} \usepackage[utf8]{inputenc}
%\usepackage[english,ukrainian,russian]{babel}
\usepackage{amssymb,amsmath}

\usepackage{tikz}
\usepackage{siunitx}
\usepackage[american,cuteinductors,smartlabels]{circuitikz}

\usepackage[backend=biber]{biblatex}
\addbibresource{error_estimation_otchet.bib}

\usepackage[]{hyperref}
\hypersetup{colorlinks=true}
\usepackage{textcomp}
\newcommand{\No}{\textnumero}
\ESKDdepartment{Федеральное агентство по образованию}
\ESKDcompany{Санкт-Петербургский государственный электротехнический университет "ЛЭТИ"}
\ESKDtitle{Пояснительная записка к Курсовой работе}
\ESKDsignature{Вариант N27}
\ESKDauthor{Храпов~М.~В.}
\ESKDchecker{Прокшин~А.~Н.}
\ESKDdocName{по дисциплине "Информатика"}
\begin{document}
\maketitle

\newpage
\tableofcontents

\newpage
\section{Цель, тема и содержание}
\textbf{Цель курсовой работы}: Уметь применять персональный компьютер и математические пакеты прикладных программ в инженерной деятельности.

\textbf{Тема курсовой работы}: Решение математических задач с использованием математического пакета
"Scilab"или "Reduce-algebra".

\textbf{Содержание курсовой работы}:
\begin{enumerate}
    \item[1.] Даны функции $f(x)=\sqrt{3}(x)+cos(x)$ и $g(x)=cos(2x+(\frac{\pi}{3})-1$
\begin{enumerate}
    \item[a)] Решить уравнение f(x)=g(x)
    \item[б)] Исследовать функцию h(x)=f(x)-g(x) на промежутке $[0;\frac{5\pi}{6}]$
\end{enumerate}
    \item[2.] Найти коэффициенты кубического сплайна, интерполирующего данные, представленные в векторах $V_y$ и $V_x$.
    \\Построить на графике функцию f(x), полученную после нахождения коэффициентов кубического сплайна.
    \\Представить графическое изображение результатов интерполяции исходных данных различными методами с использованием встроенных функций.
    \item[3.]Решить задачу оптимального распределения неоднородных ресурсов. Постановка задачи. Для изготовления n видов изделий $N_1$, $N_2$, ..., $N_n$ необходимы ресурсы m видов: трудовые, материальные, финансовые и др. Известно требуемое количество отдельного i-гo ресурса для изготовления каждого j-го изделия. Назовем эту величину нормой расхода  $c_ij$. Пусть определено количество каждого вида ресурса, которым предприятие располагает в данный момент, - $a_i$. Известна прибыль $П_i$, получаемая предприятием от изготовления каждого j-го изделия. Требуется определить, какие изделия и в каком количестве должны производиться предприятием, чтобы прибыль была максимальной.
\end{enumerate}

\newpage
\section{Исследование функции}
Первым делом найдём область определения представленных функций. Т.к. ограничений области определения в указанных функциях не наблюдается, то областью определения будет являться множество вещественных чисел ($x \epsilon R$)
\begin{enumerate}
    \item[a)] Если функции равны, тогда, приравняв выраж. к нулю получаем 
    $f(x)-g(x)=0$, и следовательно  $\sqrt{3}sin(x)+cos(x)-cos(2x+\frac{\pi}{3})+1 = 0$;
    
    Далее приводится листинг SciLab'a:
    
    $-->deff('[y]=h(x)','y1 =sqrt(3)*sin(x)+cos(x), y2 = cos((2*x) + ((\%pi)/3)) - 1, y=y1-y2')$
    
    $-->fsolve(0,h)$
    
    $ans=$
    
    - 0.5235988 => Что является значением $x$ при котором выполняется равенство $f(x)=g(x)$
    
    \item[б)]
    
    

\end{enumerate}


\end{document}
