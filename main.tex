\documentclass[russian,utf8,nocolumnxxxi,nocolumnxxxii]{eskdtext}
\usepackage[T1,T2A]{fontenc} \usepackage[utf8]{inputenc}
\usepackage[english,ukrainian,russian]{babel}
\usepackage{amssymb,amsmath}
\usepackage[shortlabels]{enumitem}
\usepackage{tikz}
\usepackage{pgfplots}
\usepackage{siunitx}
\usepackage[american,cuteinductors,smartlabels]{circuitikz}

\usepackage[backend=biber]{biblatex}
\addbibresource{error_estimation_otchet.bib}

\usepackage[]{hyperref}
\hypersetup{colorlinks=true}
\usepackage{textcomp}
\newcommand{\No}{\textnumero}
\ESKDdepartment{Федеральное агентство по образованию}
\ESKDcompany{Санкт-Петербургский государственный электротехнический университет "ЛЭТИ"}
\ESKDtitle{Пояснительная записка к Курсовой работе}
\ESKDsignature{Вариант N27}
\ESKDauthor{Храпов М.В.}
\ESKDchecker{ПрокшинА.Н.}
\ESKDdocName{по дисциплине "Информатика"}
\begin{document}
\maketitle

\newpage
\tableofcontents

\newpage

\section{Введение}

\textbf{Цель курсовой работы}: Уметь применять персональный компьютер и математические пакеты прикладных программ в инженерной деятельности.

\textbf{Тема курсовой работы}: Решение математических задач с использованием математического пакета
"Scilab" или "Reduce-algebra".

\textbf{Содержание курсовой работы}:
\begin{enumerate}
    \item[1.] Даны функции $f(x)=\sqrt{3}(x)+cos(x)$ и $g(x)=cos(2x+(\frac{\pi}{3})-1$
\begin{enumerate}
    \item[a)] Решить уравнение f(x)=g(x)
    \item[б)] Исследовать функцию h(x)=f(x)-g(x) на промежутке $[0;\frac{5\pi}{6}]$
\end{enumerate}
    \item[2.] Найти коэффициенты кубического сплайна, интерполирующего данные, представленные в векторах $V_y$ и $V_x$.
    \\Построить на графике функцию f(x), полученную после нахождения коэффициентов кубического сплайна.
    \\Представить графическое изображение результатов интерполяции исходных данных различными методами с использованием встроенных функций.
    \item[3.]Решить задачу оптимального распределения неоднородных ресурсов. Постановка задачи. Для изготовления n видов изделий $N_1$, $N_2$, ..., $N_n$ необходимы ресурсы m видов: трудовые, материальные, финансовые и др. Известно требуемое количество отдельного i-гo ресурса для изготовления каждого j-го изделия. Назовем эту величину нормой расхода  $c_ij$. Пусть определено количество каждого вида ресурса, которым предприятие располагает в данный момент, - $a_i$. Известна прибыль $П_i$, получаемая предприятием от изготовления каждого j-го изделия. Требуется определить, какие изделия и в каком количестве должны производиться предприятием, чтобы прибыль была максимальной.
\end{enumerate}

\newpage
\section{Исследование функции}

Т.к. выражений ограниченивающих область определения в указанных функциях не наблюдается, то областью определения будет являться множество вещественных чисел $(x \epsilon R)$
\begin{enumerate}
    \item[a)]Решить уравнение f(x)=g(x)
    
   Если $\sqrt{3}sin(x)+cos(x)=cos(2x+\frac{\pi}{3})-1$,
   то:
   \begin{itemize}
   \renewcommand{\labelitemi}{$\bullet$}
       \item $\sqrt{3}sin(x)+cos(x)=0$;
       
       $3tg(x)+1=0$;
       
       $tg(x)=-\frac{1}{\sqrt{3}}=-\frac{\sqrt{3}}{3}$;
       
       \item $cos(2x+\frac{\pi}{3})-1=0$;
       
       $2x+\frac{\pi}{3}=arccos 1=0$;
       
       $2x=-\frac{\pi}{3}$;

   \end{itemize}
   
 Ответ: Функции равны при $x=-\frac{\pi}{6}+\pi k,  k \epsilon Z$
    
    \item[б)] Исследовать функцию h(x)=f(x)-g(x) на промежутке $[0;\frac{5\pi}{6}]$;
    
        \textbf{План исследования}:
    \begin{enumerate}
        \item Построить график функции.
        \item Найти область определения функции. Выделить особые точки (точки разрыва). 
        \item Проверить наличие вертикальных асимптот в точках разрыва и на границах области
определения.
        \item Найти точки пересечения с осями координат.
        \item Установить, является ли функция чётной или нечётной.
        \item Определить, является ли функция периодической или нет. 
        \item Найти точки экстремума и интервалы монотонности функции.
        \item Найти точки перегиба и интервалы выпуклости-вогнутости.
         
     \end{enumerate}
        
         Преобразуем функции, получаем:
    
    $$h(x) = \sqrt{3}sin(x)+cos(x)-cos(2x+\frac{\pi}{3})+1$$
    
    График функции отображен на рисунке 1. Так-же на нём отмечен локальный максимум, находящийся в точке (x=1.04,y=4).
    
     \begin{figure}[h]
      \centering
      \caption{Функция на промежутке от 0 до $\frac{5\pi}{6}$}
    
      \begin{tikzpicture}
\draw node[below] {$0$};
\draw[->] (0,-0.1) -- (0,5) node[left] {$Y$};
\draw[->] (-0.1,0) -- (5,0) node[below] {$X$};
\draw [domain=0:2.6179938,thick,smooth,red] plot ({\x},{sqrt(3)*sin(\x r)+cos(\x r)-cos((2*\x r)+(pi/3 r))+1});% Параметр r для замены градусов на радианы
\draw (2.6179938,0) node[below] {$\frac{5\pi}{6}$};
\draw [dashed]  (0,4) node[left] {4} -- (1.047,4) -- (1.047,0) node[below] {1,047};
\end{tikzpicture}
   \end{figure}
    
    Т.к область определения функции не ограничена, следовательно точки разрыва отсутствуют.
        
        Найдём точку пересечения с Оy, приравняв x к 0.
        
        $\sqrt{3}sin(0)+cos(0)-cos(2*0+\frac{\pi}{3})+1=1.5$ - точка (0 , 1.5) => что так-же являеться левой границей области определения.
        
        Найдём точку пересечения с Оx, приравняв функцию к 0.
        
        $\sqrt{3}sin(x)+cos(x)-cos(2x+\frac{\pi}{3})+1=0$;
        
        $x=\frac{5\pi}{6}$ - точка ($\frac{5\pi}{6}$ , 0) => что так-же являеться правой границей области определения.
        
        Так-как обе границы опредилились, вертикальные асимптоты на границах функции отсутствуют.
        
        Так как на заданом интервале $h(-x) \ne h(x) \ne -h(x)$, признаками чётности/нечётности функция не обладает.
        
        Так как на заданом интервале $h(x+T) \ne h(x)$ то функция не обладает свойствами периодичности.
        
        Для поиска точек экстремума возьмём первую производную.
        
        $h'(x)=-sinx+2sin(\frac{1}{3}(6x+\pi))+\sqrt{3}cosx$ или
        
         $h'(x)=2sin(2x+\frac{\pi}{3})+2cos(x+\frac{\pi}{6})$
         
         И приравняем её нулю
         
         $$2sin(2x+\frac{\pi}{3})+2cos(x+\frac{\pi}{6})=0$$
         
          \begin{figure}[h]
      \centering
      \caption{Первая производная на промежутке от 0 до $\frac{5\pi}{6}$}
    
      \begin{tikzpicture}
\draw[->] (0,-0.1) -- (0,5) node[left] {$Y$};
\draw[->] (-0.1,0) -- (5,0) node[below] {$X$};
\draw [domain=0:2.6179938,thick,smooth,green] plot ({\x},{2*sin((2*\x r)+(pi/3 r))+2*cos((\x r)+(pi/6 r)});
\draw [dashed] (2.6179938,-2) -- (2.6179938,0) node[below right] {$\frac{5\pi}{6}$};
\draw (1.047,0) node{\textbullet};
\draw (1.047,0) node[below left] {1,047};
\draw (0,3.464) node[left] {3,464};
\end{tikzpicture}
   \end{figure}
  
  Из графика понимаем, что производная имеет одну стационарную точку (1,047).
  
  На интервале (0 , 1.047) производная положительна => функция возрастает.
  
  На оставшемся интервале (1.047,$\frac{5\pi}{6}$) производная отрицительна => функция убывает.
       
        Для нахождения точек перегиба и определения выпуклости/вогнутости необходимо найти вторую производную.
        
        $h''(x) = - \sqrt{3} \sin{\left (x \right )} - \cos{\left (x \right )} + 4 \cos{\left (2 x + \frac{\pi}{3} \right )}$ или
        
        $h''(x) = -2sin(x+\frac{\pi}{6})+4cos(2x+\frac{\pi}{3})$
        
        И приравниваем её 0
        
        $$-2sin(x+\frac{\pi}{6})+4cos(2x+\frac{\pi}{3})=0$$
        
    \begin{figure}[h]
      \centering
      \caption{Вторая производная на промежутке от 0 до $\frac{5\pi}{6}$}
    
\begin{tikzpicture}
\draw[->] (0,0) -- (0,5) node[left] {$Y$};
\draw[->] (0,0) -- (5,0) node[below] {$X$};
\draw (0,1) node[left] {1};
\draw [domain=0:2.6179938,thick,smooth,blue] plot ({\x},{-2*sin((\x r)+(pi/6 r))+4*cos((2*\x r)+(pi/3 r))});
\draw (0.1113,0) node{\textbullet};
\draw (0.1113,0) node[below left] {0,1113};
\draw[dashed] (1.047,-6) -- (1.047,0) node[above] {$\frac{\pi}{3}$};
\draw [dashed] (2.6179938,4) -- (2.6179938,0) node[below] {$\frac{5\pi}{6}$};
\end{tikzpicture}
   \end{figure}
    
На графике видно, что на интервале (0.1113,1.983) вторая производная отрицательна => на этом интервале функция выпукла. На оставшихся 2ух промежутках функция вогнута.

\end{enumerate}

\newpage
\section{Исследование кубического сплайна}

Сплайн представляет собой функцию, проходящую через жёстко заданные точки таким образом, чтобы потенциальная энергия изгибов принимала минимальное значение. Данный эффект достигается нахождением четвёртой производной данной функции, которая принемает значение 0. Исходя из этого сплайн можно представить как полином третьей степени на каждом отрезке $(xi,x_{i+1})$.

Заданы точки:
\begin{enumerate}[I]
    \item (0,4)
    \item (0.25,3.6)
    \item (1.25,4.575)
    \item (2.125,4.017)
    \item (3.25,3.833)
\end{enumerate}

Составим 8 уравнений функций:
\begin{center}

$f_1(I)=A_{10}+A_{11}I+A_{12}I^2+A_{13}I^3$

$f_1(II)=A_{10}+A_{11}II+A_{12}II^2+A_{13}II^3$

$f_2(II)=A_{20}+A_{21}II+A_{22}II^2+A_{23}II^3$

$f_2(III)=A_{20}+A_{21}III+A_{22}III^2+A_{23}III^3$

$f_3(III)=A_{30}+A_{31}III+A_{32}III^2+A_{33}III^3$

$f_3(IV)=A_{30}+A_{31}IV+A_{32}IV^2+A_{33}IV^3$

$f_4(IV)=A_{40}+A_{41}IV+A_{42}IV^2+A_{43}IV^3$

$f_4(V)=A_{40}+A_{41}V+A_{42}V^2+A_{43}V^3$

\end{center}

3 уравнения $f'$ в точках склейки:

\begin{center}
    
$A_{11}+2A_{12}II+3A_{13}II^2=A_{21}+2A_{22}II+3A_{23}II^2$

$A_{21}+2A_{22}III+3A_{23}III^2=A_{31}+2A_{32}III+3A_{33}III^2$

$A_{31}+2A_{32}IV+3A_{33}IV^2=A_{41}+2A_{42}IV+3A_{43}IV^2$
    
\end{center}

3 уравнения $f''$ в точках склейки:
\begin{center}

$2A_{12}+6A_{13}II=2A_{22}+6A_{23}II$

$2A_{22}+6A_{23}III=2A_{32}+6A_{33}III$

$2A_{32}+6A_{33}IV=2A_{42}+6A_{43}IV$

\end{center}

И, наконец, $f''=0$ в крайних точках (для свободных концов)
\begin{center}

$2A_{12}+6A_{13}I=0$

$2A_{42}+6A_{43}V=0$

\end{center}

Из получившихся 16 уравнений составим матрицу:

\makeatletter
\renewcommand*\env@matrix[1][c]{\hskip -\arraycolsep
  \let\@ifnextchar\new@ifnextchar
  \array{*\c@MaxMatrixCols #1}}
\makeatother

\resizebox{13cm}{!}{\begin{pmatrix}[0.0001cm]
$1& I& I^2& I^3& 0& 0& 0& 0& 0& 0& 0& 0& 0& 0& 0& 0 $\\
$1& II& II^2& III& 0& 0& 0& 0& 0& 0& 0& 0& 0& 0& 0& 0 $\\
$0& 1& 2II& 3II^2& 0& -1& -2II& -3II^2& 0& 0& 0& 0& 0& 0& 0& 0 $\\
$0& 0& 2& 6II& 0& 0& -2& -6II& 0& 0& 0& 0& 0& 0& 0& 0 $\\
$0& 0& 0& 0& 1& II& II^2& II^3& 0& 0& 0& 0& 0& 0& 0& 0 $\\
$0& 0& 0& 0& 1& III& III^2& III^3& 0& 0& 0& 0& 0& 0& 0& 0 $\\
$0& 0& 0& 0& 0& 1& 2III& 3III^2& 0& -1& -2III& -3III^2& 0& 0& 0& 0 $\\
$0& 0& 0& 0& 0& 0& 2& 6III& 0& 0& -2& -6III& 0& 0& 0& 0 $\\
$0& 0& 0& 0& 0& 0& 0& 0& 1& III& III^2& III^3& 0& 0& 0& 0 $\\
$0& 0& 0& 0& 0& 0& 0& 0& 1& IV& IV^2& IV^3& 0& 0& 0& 0 $\\
$0& 0& 0& 0& 0& 0& 0& 0& 0& 1& 2IV& 3IV^2& 0& -1& -2IV& -3IV^2 $\\
$0& 0& 0& 0& 0& 0& 0& 0& 0& 0& 2& 6IV& 0& 0& -2& -6IV $\\
$0& 0& 0& 0& 0& 0& 0& 0& 0& 0& 0& 0& 1& IV& IV^2& IV^3 $\\
$0& 0& 0& 0& 0& 0& 0& 0& 0& 0& 0& 0& 1& V& V^2& V^3 $\\
$0& 0& 2& 6I& 0& 0& 0& 0& 0& 0& 0& 0& 0& 0& 0& 0 $\\
$0& 0& 0& 0& 0& 0& 0& 0& 0& 0& 0& 0& 0& 0& 2& 6V $
\end{pmatrix}}*\resizebox{1.14cm}{!}{\begin{pmatrix}
$A_{10}$ \\
$A_{11}$ \\
$A_{12}$ \\
$A_{13}$ \\
$A_{20}$ \\
$A_{21}$ \\
$A_{22}$ \\
$A_{23}$ \\
$A_{30}$ \\
$A_{31}$ \\
$A_{32}$ \\
$A_{33}$ \\
$A_{40}$ \\
$A_{41}$ \\
$A_{42}$ \\
$A_{43}$ 
\end{pmatrix}}=\resizebox{1.06cm}{!}{\begin{pmatrix}
I \\
II \\
0 \\
0 \\
II \\
III \\
0 \\
0 \\
III \\
IV \\
0 \\
0 \\
IV \\
V \\
0 \\
0
\end{pmatrix}}


Её решение представляется данной матрицой-вектором:
\resizebox{1.7cm}{!}{\begin{pmatrix}
4\\
-1.956\\
0\\
5.965\\
4.127\\
-3.475\\
6.077\\
-2.408\\
-0.2\\
8.303\\
-4.461\\
0.7\\
-1.165\\
9.67\\
-5.104\\
0.801
\end{pmatrix}}

\newpage
В итоге система уравнений для сплайна выглядит так:

\begin{equation*}
f(x) =
 \begin{cases}
  $f_1(x)=5.965x^3-1.956x+4$
   \\
  $f_2(x)=-2.408x^3+6.077x^2-3.475x+4.127$
   \\
  $f_3(x)=0.7x^3-4.46x^2+8.303x-0.2$
  \\
  $f_4(x)=0.801x^3-5.104x^2+9.67x-1.165$
 \end{cases}
\end{equation*}

 \begin{figure}[h]
      \centering
      \caption{Сплайн и его точки}
\begin{tikzpicture}[xscale=4]
\draw [<->] (0,5.5) node[left] {$Y$} -- (0,0) node[below] {$0$} -- (3.5,0) node[below] {$X$};
\draw[domain=0:0.25, smooth, dashed] plot ({\x},{(5.956*((\x)*(\x)*(\x)))-(1.956*(\x))+4});
\draw[domain=0.25:1.25, smooth, green] plot ({\x},{(-2.408*((\x)*(\x)*(\x)))+(6.077*((\x)*(\x)))-(3.475*(\x))+4.127});
\draw[domain=1.25:2.125, smooth, dashed] plot ({\x},{(0.7*((\x)*(\x)*(\x)))-(4.46*((\x)*(\x)))+(8.303*(\x))-0.2});
\draw[domain=2.125:3.25, smooth, blue] plot ({\x},{(0.801*((\x)*(\x)*(\x)))-(5.104*((\x)*(\x)))+(9.67*(\x))-1.165});
\draw (0,4) node{\textbullet} node[left] {(0,4)};
\draw (0.25,3.6) node{\textbullet} node[below] {(0.25,3.6)};
\draw (1.25,4.575) node{\textbullet} node[above] {(1.25,4.575)};
\draw (2.125,4.017) node{\textbullet} node[below left] {(2.125,4.017)};
\draw (3.25,3.833) node{\textbullet} node[above] {(3.25,3.833)};
\end{tikzpicture}
   \end{figure}
   
   \subsection{Расчёт погрешности интерволяции при помощи этмитовых кубических сплайнов.}
   
   Оценка интерволяции эрмитовым кубическим сплайном вида
   
   $$|S_3^{(r)}(x)-f^{(r)}(x)|\leqslant R_r,r=0,1,2,3$$
   
   Представляется для гладкой функции:
   
   $$|S_3^{(r)}(x)-f^{(r)}(x)|\leqslant \frac{1}{384}\overline{h}^4|f^{IV}(x)|$$
   
   где: \overline{h}=\bigg| x \textsubscript {точка, в кот. вычисл. погрешность} -x \textsubscript {ближайшее i} \bigg|
  
  \newpage
  
   Для нахождения коэффициентов импользуем метод разделённой разности:
   
   \resizebox{16cm}{!}{$$P(x) = A_0+A_1(x-x_0)+A_2(x-x_0)(x-x_1)+A_3(x-x_0)(x-x_1)(x-x_2)+A_4(x-x_0)(x-x_1)(x-x_2)(x-x_3)$$}
   
   
\vspace{20pt}    

\centering где:
   
   $$A_0=f(x_0)$$
   \vspace{20pt}
   $$A_1=\frac{f(x_1)}{x_1-x_0}+\frac{f(x_0)}{x_0-x_1}$$
   \vspace{20pt}
   $$A_2=\frac{f(x_2)}{(x_2-x_1) \cdot (x_2-x_0)} + \frac{f(x_1)}{(x_1-x_2) \cdot (x_1-x_0)} + \frac{f(x_0)}{(x_0-x_2) \cdot (x_0-x_1)}$$
   \vspace{20pt}
   $$A_3=\frac{f(x_3)}{(x_3-x_2) \cdot (x_3-x_1) \cdot (x_3-x_0)} + \frac{f(x_2)}{(x_2-x_0) \cdot (x_2-x_1) \cdot (x_2-x_3)} +$$ 
   $$+ \frac{f(x_1)}{(x_1-x_0) \cdot (x_1-x_2) \cdot (x_1-x_3)} + \frac{f(x_0)}{(x_0-x_1) \cdot (x_0-x_2) \cdot (x_0-x_3)}$$
   \vspace{20pt}
   $$A_4=\frac{f(x_4)}{(x_4-x_3) \cdot (x_4-x_2) \cdot (x_4-x_1) \cdot (x_4-x_0)} +$$
   $$+ \frac{f(x_3)}{(x_3-x_4) \cdot (x_3-x_2) \cdot (x_3-x_1) \cdot (x_3-x_0)} +$$
   $$+ \frac{f(x_2)}{(x_2-x_4) \cdot (x_2-x_3) \cdot (x_2-x_1) \cdot (x_2-x_0)} +$$
   $$+ \frac{f(x_1)}{(x_1-x_4) \cdot (x_1-x_3) \cdot (x_1-x_2) \cdot (x_1-x_0)} +$$
   $$+ \frac{f(x_0)}{(x_0-x_4) \cdot (x_0-x_3) \cdot (x_0-x_2) \cdot (x_0-x_1)}$$
   
   \vspace{20pt}
   
   $A_0=4$;$A_1=-1.6$;$A_2=2.06$;$A_3=-1.374$;$A_4=0.5353547$
   
   \newpage
   
   После чего составим полином Ньютона.
   
   $$f(x)=\frac{68480x^4-424112x^3+767168x^2-371155x+512000}{128000}$$
   
   По заданию нам нужно отыскать значение функции в точке 1.2:
   
   $f(1.2)=4.534925875$
   
   Так-же требуется найти погрешность в точке 2.2. Формула для рассчёта погрешности указана в начале секции, и дублирована здесь.
   
    $$|S_3^{(r)}(x)-f^{(r)}(x)|\leqslant \frac{1}{384}\overline{h}^4|f^{IV}(x)|$$
    
    Найдя $f^{IV}(x) = \frac{312}{25}$ и подставив в формулу вместе с h=0.075 (2.2-2.125=0.075), получим погрешность равную $1.057 \cdot 10^{-6}$
    
    \newpage
    \section{Задача оптимального распределения неоднородных ресурсов}
    
   
        \begin{table}[h]
        \centering
     \caption{Условия поставленой задачи}
\begin{tabular}[c]{|*{6}{c|}}
\hline
Исп. рес-ы & Изд_1 & Изд_2 & Изд_3 & Изд_4 & Наличие\\
\hline
Труд. & 3 & 5 & 5 & 3 & 11\\
\hline
Матер. & 4 & 5 & 8 & 5 & 8\\
\hline
Фин. & 5 & 6 & 4 & 8 & 26\\
\hline
Profit & 40 & 50 & 25 & 25\\
\hline
\end{tabular}
\end{table}

Для нахождения оптимального решения воспользуемся функцией \textit{linpro} пакета \textit{SciLab}, где \textit{"p"} - коэф. при неизвестных целевой $f$, \textit{"C"} - матрица неизвествых системы ограничений, а \textit{"b"} содержит свободные члены (\textit {"ci"} и \textit{"cs"} - соответственно нижняя и верхняя границы переменных).

 \vspace{20pt}
 
Система ограничений выглядит следующим образом.

\begin{equation*}
\begin{cases}
  3x_1+5x_2+5x_3+3x_4 \leqslant 11
   \\
  4x_1+5x_2+8x_3+5x_4 \leqslant 8
   \\
  5x_1+6x_2+4x_3+4x_4 \leqslant 26
  \end{cases}
\end{equation*}

\vspace{20pt}

Составляем:
  
   \textit{f_{max}} = $40x_1+50x_2+25x_3+25x_4$
   
   \vspace{20pt}
   
   \textit{C}= \begin{pmatrix}
   3 \ 5 \ 5 \ 3 \\
   4 \ 5 \ 8 \ 5 \\
   5 \ 6 \ 4 \ 4 
   \end{pmatrix}  \textit{b}= \begin{pmatrix}
   11 \\
   8 \\
   26
   \end{pmatrix} \textit{p}= \begin{pmatrix}
   40 \\
   50 \\
   25 \\
   25
   \end{pmatrix}
   
  В итоге было рассчитаны значения, при которых прибыль максимальна:
  
  \textit{f_{max}} = 80 
  
  \textit{larg} = (0;0;55;25;0;10;0)
  
  \textit{x} = (0;1.6;0;0)
  
  Ответ: Прибыль максимальна при производстве 1.6 единиц изделия №2.
   
  
\newpage
\begin{thebibliography}{}
    \bibitem .Ю.С. Завьялов. \textit{Методы сплайн-функций.} М.Наука, 1980.
    \bibitem .Калиткин. \textit{Численные методы.} М.,Мир, 1980.
    \bibitem  .\textit{"Разделённая разность".} https://ru.wikipedia.org/wiki/Разделённая\_разность
    \bibitem .\textit{Викиучебник- "Scilab/Графики".}https://ru.wikibooks.org/wiki/Scilab/Графики
    \bibitem .Андриевский А.Б., Андриевский Б.Р., Капитонов А.А., Фрадков А.Л. \textit{РЕШЕНИЕ ИНЖЕНЕРНЫХ ЗАДАЧ В SCILAB} - Санкт-Петербург: НИУ ИТМО, 2013. - 97 с.
    
\end{thebibliography}

\end{document}
